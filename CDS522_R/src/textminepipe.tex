% Options for packages loaded elsewhere
\PassOptionsToPackage{unicode}{hyperref}
\PassOptionsToPackage{hyphens}{url}
%
\documentclass[
]{article}
\usepackage{lmodern}
\usepackage{amssymb,amsmath}
\usepackage{ifxetex,ifluatex}
\ifnum 0\ifxetex 1\fi\ifluatex 1\fi=0 % if pdftex
  \usepackage[T1]{fontenc}
  \usepackage[utf8]{inputenc}
  \usepackage{textcomp} % provide euro and other symbols
\else % if luatex or xetex
  \usepackage{unicode-math}
  \defaultfontfeatures{Scale=MatchLowercase}
  \defaultfontfeatures[\rmfamily]{Ligatures=TeX,Scale=1}
\fi
% Use upquote if available, for straight quotes in verbatim environments
\IfFileExists{upquote.sty}{\usepackage{upquote}}{}
\IfFileExists{microtype.sty}{% use microtype if available
  \usepackage[]{microtype}
  \UseMicrotypeSet[protrusion]{basicmath} % disable protrusion for tt fonts
}{}
\makeatletter
\@ifundefined{KOMAClassName}{% if non-KOMA class
  \IfFileExists{parskip.sty}{%
    \usepackage{parskip}
  }{% else
    \setlength{\parindent}{0pt}
    \setlength{\parskip}{6pt plus 2pt minus 1pt}}
}{% if KOMA class
  \KOMAoptions{parskip=half}}
\makeatother
\usepackage{xcolor}
\IfFileExists{xurl.sty}{\usepackage{xurl}}{} % add URL line breaks if available
\IfFileExists{bookmark.sty}{\usepackage{bookmark}}{\usepackage{hyperref}}
\hypersetup{
  pdftitle={Text Mining Pipeline},
  hidelinks,
  pdfcreator={LaTeX via pandoc}}
\urlstyle{same} % disable monospaced font for URLs
\usepackage[margin=1in]{geometry}
\usepackage{color}
\usepackage{fancyvrb}
\newcommand{\VerbBar}{|}
\newcommand{\VERB}{\Verb[commandchars=\\\{\}]}
\DefineVerbatimEnvironment{Highlighting}{Verbatim}{commandchars=\\\{\}}
% Add ',fontsize=\small' for more characters per line
\usepackage{framed}
\definecolor{shadecolor}{RGB}{248,248,248}
\newenvironment{Shaded}{\begin{snugshade}}{\end{snugshade}}
\newcommand{\AlertTok}[1]{\textcolor[rgb]{0.94,0.16,0.16}{#1}}
\newcommand{\AnnotationTok}[1]{\textcolor[rgb]{0.56,0.35,0.01}{\textbf{\textit{#1}}}}
\newcommand{\AttributeTok}[1]{\textcolor[rgb]{0.77,0.63,0.00}{#1}}
\newcommand{\BaseNTok}[1]{\textcolor[rgb]{0.00,0.00,0.81}{#1}}
\newcommand{\BuiltInTok}[1]{#1}
\newcommand{\CharTok}[1]{\textcolor[rgb]{0.31,0.60,0.02}{#1}}
\newcommand{\CommentTok}[1]{\textcolor[rgb]{0.56,0.35,0.01}{\textit{#1}}}
\newcommand{\CommentVarTok}[1]{\textcolor[rgb]{0.56,0.35,0.01}{\textbf{\textit{#1}}}}
\newcommand{\ConstantTok}[1]{\textcolor[rgb]{0.00,0.00,0.00}{#1}}
\newcommand{\ControlFlowTok}[1]{\textcolor[rgb]{0.13,0.29,0.53}{\textbf{#1}}}
\newcommand{\DataTypeTok}[1]{\textcolor[rgb]{0.13,0.29,0.53}{#1}}
\newcommand{\DecValTok}[1]{\textcolor[rgb]{0.00,0.00,0.81}{#1}}
\newcommand{\DocumentationTok}[1]{\textcolor[rgb]{0.56,0.35,0.01}{\textbf{\textit{#1}}}}
\newcommand{\ErrorTok}[1]{\textcolor[rgb]{0.64,0.00,0.00}{\textbf{#1}}}
\newcommand{\ExtensionTok}[1]{#1}
\newcommand{\FloatTok}[1]{\textcolor[rgb]{0.00,0.00,0.81}{#1}}
\newcommand{\FunctionTok}[1]{\textcolor[rgb]{0.00,0.00,0.00}{#1}}
\newcommand{\ImportTok}[1]{#1}
\newcommand{\InformationTok}[1]{\textcolor[rgb]{0.56,0.35,0.01}{\textbf{\textit{#1}}}}
\newcommand{\KeywordTok}[1]{\textcolor[rgb]{0.13,0.29,0.53}{\textbf{#1}}}
\newcommand{\NormalTok}[1]{#1}
\newcommand{\OperatorTok}[1]{\textcolor[rgb]{0.81,0.36,0.00}{\textbf{#1}}}
\newcommand{\OtherTok}[1]{\textcolor[rgb]{0.56,0.35,0.01}{#1}}
\newcommand{\PreprocessorTok}[1]{\textcolor[rgb]{0.56,0.35,0.01}{\textit{#1}}}
\newcommand{\RegionMarkerTok}[1]{#1}
\newcommand{\SpecialCharTok}[1]{\textcolor[rgb]{0.00,0.00,0.00}{#1}}
\newcommand{\SpecialStringTok}[1]{\textcolor[rgb]{0.31,0.60,0.02}{#1}}
\newcommand{\StringTok}[1]{\textcolor[rgb]{0.31,0.60,0.02}{#1}}
\newcommand{\VariableTok}[1]{\textcolor[rgb]{0.00,0.00,0.00}{#1}}
\newcommand{\VerbatimStringTok}[1]{\textcolor[rgb]{0.31,0.60,0.02}{#1}}
\newcommand{\WarningTok}[1]{\textcolor[rgb]{0.56,0.35,0.01}{\textbf{\textit{#1}}}}
\usepackage{longtable,booktabs}
% Correct order of tables after \paragraph or \subparagraph
\usepackage{etoolbox}
\makeatletter
\patchcmd\longtable{\par}{\if@noskipsec\mbox{}\fi\par}{}{}
\makeatother
% Allow footnotes in longtable head/foot
\IfFileExists{footnotehyper.sty}{\usepackage{footnotehyper}}{\usepackage{footnote}}
\makesavenoteenv{longtable}
\usepackage{graphicx,grffile}
\makeatletter
\def\maxwidth{\ifdim\Gin@nat@width>\linewidth\linewidth\else\Gin@nat@width\fi}
\def\maxheight{\ifdim\Gin@nat@height>\textheight\textheight\else\Gin@nat@height\fi}
\makeatother
% Scale images if necessary, so that they will not overflow the page
% margins by default, and it is still possible to overwrite the defaults
% using explicit options in \includegraphics[width, height, ...]{}
\setkeys{Gin}{width=\maxwidth,height=\maxheight,keepaspectratio}
% Set default figure placement to htbp
\makeatletter
\def\fps@figure{htbp}
\makeatother
\setlength{\emergencystretch}{3em} % prevent overfull lines
\providecommand{\tightlist}{%
  \setlength{\itemsep}{0pt}\setlength{\parskip}{0pt}}
\setcounter{secnumdepth}{-\maxdimen} % remove section numbering

\title{Text Mining Pipeline}
\author{}
\date{\vspace{-2.5em}}

\begin{document}
\maketitle

A simple R project adapted for self-study purpose.

\hypertarget{overview}{%
\subsection{Overview}\label{overview}}

\begin{itemize}
\tightlist
\item
  Section 1: Create R Project
\item
  Section 2: Loading Data into R

  \begin{itemize}
  \tightlist
  \item
    2-1: Text File
  \end{itemize}
\item
  Section 3: Text Pre-processing
\item
  Section 4: Text Normalization

  \begin{itemize}
  \tightlist
  \item
    4-1: Stemming
  \item
    4-2: Lemmatization
  \end{itemize}
\item
  Section 5: Text Representation

  \begin{itemize}
  \tightlist
  \item
    5-1: Document Term Matrix (DTM)
  \end{itemize}
\item
  Section 6: Text Mining

  \begin{itemize}
  \tightlist
  \item
    6-1: Word Frequency
  \item
    6-2: Term Correlation
  \end{itemize}
\item
  Section 7: Simple Graphics

  \begin{itemize}
  \tightlist
  \item
    7-1: Histogram
  \item
    7-2: Word Cloud
  \end{itemize}
\end{itemize}

\begin{center}\rule{0.5\linewidth}{0.5pt}\end{center}

\hypertarget{section-1-create-r-project}{%
\section{Section 1: Create R Project}\label{section-1-create-r-project}}

\hypertarget{create-a-new-project-in-rstudio}{%
\subsubsection{Create a New Project in
RStudio}\label{create-a-new-project-in-rstudio}}

The steps to create a new project in R is specified as follows:

\begin{enumerate}
\def\labelenumi{\arabic{enumi}.}
\tightlist
\item
  Click the \textbf{File \textgreater{} New Project} from the top menu.
\item
  Click \textbf{New Directory}.
\item
  Click \textbf{New Project}.
\item
  Enter the directory name to store your project,
  e.g.~``simpleTextMining''.
\item
  \emph{(optional)} Place the project under your selected subdirectory.
\item
  Click \textbf{Create Project} button.
\end{enumerate}

\hypertarget{create-subdirectories-for-project}{%
\subsubsection{Create Subdirectories for
Project}\label{create-subdirectories-for-project}}

The subfolders created for this project, together with the files in
these subfolders are as listed below:

\begin{enumerate}
\def\labelenumi{\arabic{enumi}.}
\tightlist
\item
  \textbf{doc}: text documents associated with the project.
\item
  \textbf{data}: raw data and metadata.
\item
  \textbf{output}: files generated during cleanup and analysis.
\item
  \textbf{src}: source for the project's scripts and programs
\item
  \textbf{bin}: programs brought in from elsewhere or compiled locally
\end{enumerate}

Finally, all files are named to reflect their content or function.

\hypertarget{create-r-script-file}{%
\subsubsection{Create R Script File}\label{create-r-script-file}}

Create an R Script file to save your codes.

Write the code you want to run directly in an .R script file, and then
running the selected lines (keyboard shortcut: \textbf{Ctrl + Enter}) in
the interactive R console.

Save your R Script file in the \textbf{src} folder.

\hypertarget{check-working-directory}{%
\subsubsection{Check Working Directory}\label{check-working-directory}}

When you are working with R using a Project environment, the workspace
will be automatically loaded when you open the project. To change the
working directory, call \texttt{setwd()}.

\begin{Shaded}
\begin{Highlighting}[]
\CommentTok{# Get current working directory}
\KeywordTok{getwd}\NormalTok{()}
\end{Highlighting}
\end{Shaded}

\begin{verbatim}
## [1] "C:/Users/Darren Lee/Jupyter_Notebook/MS_DSA/CDS522_2021/CDS522_R/src"
\end{verbatim}

\hypertarget{load-r-packages}{%
\subsubsection{Load R Packages}\label{load-r-packages}}

Call \texttt{library(\{package\_name\})} to load the package.

\begin{Shaded}
\begin{Highlighting}[]
\CommentTok{# Load package}
\KeywordTok{library}\NormalTok{(tm)}
\end{Highlighting}
\end{Shaded}

\begin{verbatim}
## Warning: package 'tm' was built under R version 3.6.3
\end{verbatim}

\begin{verbatim}
## Loading required package: NLP
\end{verbatim}

\begin{verbatim}
## Warning: package 'NLP' was built under R version 3.6.3
\end{verbatim}

\begin{center}\rule{0.5\linewidth}{0.5pt}\end{center}

\hypertarget{section-2-loading-data-into-r}{%
\section{Section 2: Loading Data into
R}\label{section-2-loading-data-into-r}}

\hypertarget{load-and-view-data-in-r}{%
\subsubsection{Load and View Data in R}\label{load-and-view-data-in-r}}

\begin{Shaded}
\begin{Highlighting}[]
\CommentTok{# Create Corpus from .txt}
\NormalTok{docs <-}\StringTok{ }\KeywordTok{Corpus}\NormalTok{(}\KeywordTok{DirSource}\NormalTok{(}\StringTok{"../data"}\NormalTok{))}

\CommentTok{# View Corpus Information}
\KeywordTok{print}\NormalTok{(docs)}
\end{Highlighting}
\end{Shaded}

\begin{verbatim}
## <<SimpleCorpus>>
## Metadata:  corpus specific: 1, document level (indexed): 0
## Content:  documents: 10
\end{verbatim}

\begin{Shaded}
\begin{Highlighting}[]
\KeywordTok{print}\NormalTok{(}\KeywordTok{summary}\NormalTok{(docs))}
\end{Highlighting}
\end{Shaded}

\begin{verbatim}
##                        Length Class             Mode
## childrenstories_01.txt 2      PlainTextDocument list
## childrenstories_02.txt 2      PlainTextDocument list
## childrenstories_03.txt 2      PlainTextDocument list
## childrenstories_04.txt 2      PlainTextDocument list
## childrenstories_05.txt 2      PlainTextDocument list
## childrenstories_06.txt 2      PlainTextDocument list
## childrenstories_07.txt 2      PlainTextDocument list
## childrenstories_08.txt 2      PlainTextDocument list
## childrenstories_09.txt 2      PlainTextDocument list
## childrenstories_10.txt 2      PlainTextDocument list
\end{verbatim}

\begin{Shaded}
\begin{Highlighting}[]
\KeywordTok{cat}\NormalTok{(}\StringTok{"Welcome to this simple text mining project"}\NormalTok{)}
\end{Highlighting}
\end{Shaded}

\begin{verbatim}
## Welcome to this simple text mining project
\end{verbatim}

\begin{Shaded}
\begin{Highlighting}[]
\KeywordTok{require}\NormalTok{(tm)}
\end{Highlighting}
\end{Shaded}

\hypertarget{inspect-document-contents}{%
\subsubsection{Inspect Document
Contents}\label{inspect-document-contents}}

You can examine the contents of a particular document (e.g.~the first
document)

\begin{Shaded}
\begin{Highlighting}[]
\CommentTok{# Inspect a particular document (e.g. the 1st doc)}
\KeywordTok{writeLines}\NormalTok{(}\KeywordTok{as.character}\NormalTok{(docs[[}\DecValTok{1}\NormalTok{]]))}
\end{Highlighting}
\end{Shaded}

\begin{verbatim}
## Was it just another game of hide and seek? No. It was not. First she fell into a deep, dark hole in the ground and then they found a treasure. Did it end there? No! It did not. Read more about this thrilling adventure of Sally and friends in this free illustrated kids’ book. The fun never ends when Sally’s around!
\end{verbatim}

\begin{center}\rule{0.5\linewidth}{0.5pt}\end{center}

\hypertarget{section-3-text-pre-processing}{%
\section{Section 3: Text
Pre-processing}\label{section-3-text-pre-processing}}

\hypertarget{data-cleaning-using-tm-package}{%
\subsubsection{\texorpdfstring{Data Cleaning using \texttt{tm}
Package}{Data Cleaning using tm Package}}\label{data-cleaning-using-tm-package}}

Data cleaning is an important step in text analysis.

It could take up to few cycles to achieve a mature cleaning pipeline as
new issues are often found during the process of cleaning.

\texttt{tm} package offers a number of text transformation functions.
Call \texttt{getTransformation()} to list these transformation
functions.

\begin{Shaded}
\begin{Highlighting}[]
\CommentTok{# checkout tm package transformation functions}
\KeywordTok{getTransformations}\NormalTok{()}
\end{Highlighting}
\end{Shaded}

\begin{verbatim}
## [1] "removeNumbers"     "removePunctuation" "removeWords"      
## [4] "stemDocument"      "stripWhitespace"
\end{verbatim}

\hypertarget{create-a-new-function-tospace}{%
\subsubsection{\texorpdfstring{Create A New Function:
\texttt{toSpace}}{Create A New Function: toSpace}}\label{create-a-new-function-tospace}}

\begin{Shaded}
\begin{Highlighting}[]
\CommentTok{# Create toSpace content transformer}
\NormalTok{toSpace <-}\StringTok{ }\KeywordTok{content_transformer}\NormalTok{(}
  \ControlFlowTok{function}\NormalTok{(x, pattern) \{}
    \KeywordTok{return}\NormalTok{ (}\KeywordTok{gsub}\NormalTok{(pattern, }\StringTok{" "}\NormalTok{, x))}
\NormalTok{  \}}
\NormalTok{)}
\end{Highlighting}
\end{Shaded}

The description of the parameters and variables in the new function:
\texttt{toSpace}:

\begin{itemize}
\tightlist
\item
  \texttt{gsub()}: replace all occurrences of a pattern.
\item
  \texttt{pattern}: a pattern to search for (assumed to be a regex)

  \begin{itemize}
  \tightlist
  \item
    an additional argument \texttt{fixed\ =\ TRUE} can be specified to
    look for a pattern without using regex.
  \end{itemize}
\item
  \texttt{replacement}: a character string to replace the occurrence (or
  occurrences for \texttt{gsub}) of pattern.

  \begin{itemize}
  \tightlist
  \item
    Here, \texttt{replacement\ =\ "\ "}
  \end{itemize}
\item
  \texttt{x}: a character vector to search for pattern. Each element
  will be searched separately.
\end{itemize}

\hypertarget{before-transformation}{%
\subsubsection{Before Transformation}\label{before-transformation}}

\begin{Shaded}
\begin{Highlighting}[]
\CommentTok{# select a doc}
\NormalTok{docIndex <-}\StringTok{ }\DecValTok{3}

\CommentTok{# before transformation}
\KeywordTok{writeLines}\NormalTok{(}\KeywordTok{as.character}\NormalTok{(docs[[docIndex]]))}
\end{Highlighting}
\end{Shaded}

\begin{verbatim}
## Love shines through this great illustrated kids’ book . Read how a little girl makes chores fun and easy to do. A  fantastic addition to your little one’s free bed time story collection.
\end{verbatim}

\hypertarget{replace-special-punctuation-with-space}{%
\subsubsection{Replace Special Punctuation with
Space}\label{replace-special-punctuation-with-space}}

Call the helper function \texttt{toSpace()} as previously defined.

\begin{Shaded}
\begin{Highlighting}[]
\CommentTok{# eliminate hyphen using toSpace content transformation}
\NormalTok{docs <-}\StringTok{ }\KeywordTok{tm_map}\NormalTok{(docs, toSpace, }\StringTok{"-"}\NormalTok{)}
\KeywordTok{writeLines}\NormalTok{(}\KeywordTok{as.character}\NormalTok{(docs[[docIndex]]))}
\end{Highlighting}
\end{Shaded}

\begin{verbatim}
## Love shines through this great illustrated kids’ book . Read how a little girl makes chores fun and easy to do. A  fantastic addition to your little one’s free bed time story collection.
\end{verbatim}

\hypertarget{remove-punctuation}{%
\subsubsection{Remove Punctuation}\label{remove-punctuation}}

\begin{Shaded}
\begin{Highlighting}[]
\CommentTok{# apply removePunctuation}
\NormalTok{docs <-}\StringTok{ }\KeywordTok{tm_map}\NormalTok{(docs, removePunctuation)}
\KeywordTok{writeLines}\NormalTok{(}\KeywordTok{as.character}\NormalTok{(docs[[docIndex]]))}
\end{Highlighting}
\end{Shaded}

\begin{verbatim}
## Love shines through this great illustrated kids’ book  Read how a little girl makes chores fun and easy to do A  fantastic addition to your little one’s free bed time story collection
\end{verbatim}

\hypertarget{convert-to-lower-case}{%
\subsubsection{Convert to Lower Case}\label{convert-to-lower-case}}

\begin{Shaded}
\begin{Highlighting}[]
\CommentTok{# convert corpus to lower case}
\NormalTok{docs <-}\StringTok{ }\KeywordTok{tm_map}\NormalTok{(docs, }\KeywordTok{content_transformer}\NormalTok{(tolower))}
\KeywordTok{writeLines}\NormalTok{(}\KeywordTok{as.character}\NormalTok{(docs[[docIndex]]))}
\end{Highlighting}
\end{Shaded}

\begin{verbatim}
## love shines through this great illustrated kids’ book  read how a little girl makes chores fun and easy to do a  fantastic addition to your little one’s free bed time story collection
\end{verbatim}

\hypertarget{remove-numbers}{%
\subsubsection{Remove Numbers}\label{remove-numbers}}

\begin{Shaded}
\begin{Highlighting}[]
\CommentTok{# remove digits in corpus}
\NormalTok{docs <-}\StringTok{ }\KeywordTok{tm_map}\NormalTok{(docs, removeNumbers)}
\KeywordTok{writeLines}\NormalTok{(}\KeywordTok{as.character}\NormalTok{(docs[[docIndex]]))}
\end{Highlighting}
\end{Shaded}

\begin{verbatim}
## love shines through this great illustrated kids’ book  read how a little girl makes chores fun and easy to do a  fantastic addition to your little one’s free bed time story collection
\end{verbatim}

\hypertarget{remove-stopwords}{%
\subsubsection{Remove Stopwords}\label{remove-stopwords}}

Example of stopwords recognized by the \texttt{tm} package includes:

\begin{itemize}
\tightlist
\item
  articles: a, an, the
\item
  conjuctions: and, or, but
\item
  common verbs: is
\item
  qualifiers: yet, however
\end{itemize}

\begin{Shaded}
\begin{Highlighting}[]
\CommentTok{# remove stopwords using the standard list in tm}
\NormalTok{docs <-}\StringTok{ }\KeywordTok{tm_map}\NormalTok{(docs, removeWords, }\KeywordTok{stopwords}\NormalTok{(}\StringTok{"english"}\NormalTok{))}
\KeywordTok{writeLines}\NormalTok{(}\KeywordTok{as.character}\NormalTok{(docs[[docIndex]]))}
\end{Highlighting}
\end{Shaded}

\begin{verbatim}
## love shines   great illustrated kids’ book  read   little girl makes chores fun  easy     fantastic addition   little one’s free bed time story collection
\end{verbatim}

\hypertarget{strip-extra-whitespace}{%
\subsubsection{Strip Extra Whitespace}\label{strip-extra-whitespace}}

\begin{Shaded}
\begin{Highlighting}[]
\CommentTok{# remove whitespace optional to remove extra whitespace}
\NormalTok{docs <-}\StringTok{ }\KeywordTok{tm_map}\NormalTok{(docs, stripWhitespace)}
\KeywordTok{writeLines}\NormalTok{(}\KeywordTok{as.character}\NormalTok{(docs[[docIndex]]))}
\end{Highlighting}
\end{Shaded}

\begin{verbatim}
## love shines great illustrated kids’ book read little girl makes chores fun easy fantastic addition little one’s free bed time story collection
\end{verbatim}

\begin{center}\rule{0.5\linewidth}{0.5pt}\end{center}

\hypertarget{section-4-text-normalization}{%
\section{Section 4: Text
Normalization}\label{section-4-text-normalization}}

\hypertarget{stemming}{%
\subsection{4-1: Stemming}\label{stemming}}

\textbf{Stemming} is the process of reducing related words to their
common root. For example:

\begin{quote}
offer, offered, offering \(\rightarrow\) offer
\end{quote}

Simple stemming algorithms (in \texttt{tm} package) simply chop off the
ends of the words.

To perform stemming, pass function \texttt{stemDocument()} (from
\texttt{SnowballC} package) to \texttt{tm\_map()} of \texttt{tm}
package.

\begin{Shaded}
\begin{Highlighting}[]
\KeywordTok{library}\NormalTok{(SnowballC)}
\end{Highlighting}
\end{Shaded}

\begin{verbatim}
## Warning: package 'SnowballC' was built under R version 3.6.3
\end{verbatim}

\begin{Shaded}
\begin{Highlighting}[]
\CommentTok{# duplicate object for testing}
\NormalTok{docs.stem <-}\StringTok{ }\NormalTok{docs}

\CommentTok{# stem the corpus}
\NormalTok{docs.stem <-}\StringTok{ }\KeywordTok{tm_map}\NormalTok{(docs.stem, stemDocument)}
\KeywordTok{writeLines}\NormalTok{(}\KeywordTok{as.character}\NormalTok{(docs.stem[[}\DecValTok{2}\NormalTok{]]))}
\end{Highlighting}
\end{Shaded}

\begin{verbatim}
## read warm tale camaraderi affect set wild beauti savannah free illustr kid book ginger giraff use long neck save anim blaze forest fire follow jungl path meet yet anoth adventur
\end{verbatim}

\hypertarget{lemmatization}{%
\subsection{4-2: Lemmatization}\label{lemmatization}}

\textbf{Lemmatization} is the process of grouping together the inflected
forms of a word. It is much more sophisticated as compared to stemming.

The resulting \emph{lemma} can be analyzed as a single item.

To perform lemmatization, pass function \texttt{lemmatize\_string()}
(from \texttt{textstem} package) to \texttt{tm\_map()} of \texttt{tm}
package.

\begin{Shaded}
\begin{Highlighting}[]
\CommentTok{# Lemmatization}
\CommentTok{# load textstem package}
\KeywordTok{library}\NormalTok{(textstem)}
\end{Highlighting}
\end{Shaded}

\begin{verbatim}
## Warning: package 'textstem' was built under R version 3.6.3
\end{verbatim}

\begin{verbatim}
## Loading required package: koRpus.lang.en
\end{verbatim}

\begin{verbatim}
## Warning: package 'koRpus.lang.en' was built under R version 3.6.3
\end{verbatim}

\begin{verbatim}
## Loading required package: koRpus
\end{verbatim}

\begin{verbatim}
## Warning: package 'koRpus' was built under R version 3.6.3
\end{verbatim}

\begin{verbatim}
## Loading required package: sylly
\end{verbatim}

\begin{verbatim}
## Warning: package 'sylly' was built under R version 3.6.3
\end{verbatim}

\begin{verbatim}
## For information on available language packages for 'koRpus', run
## 
##   available.koRpus.lang()
## 
## and see ?install.koRpus.lang()
\end{verbatim}

\begin{verbatim}
## 
## Attaching package: 'koRpus'
\end{verbatim}

\begin{verbatim}
## The following object is masked from 'package:tm':
## 
##     readTagged
\end{verbatim}

\begin{Shaded}
\begin{Highlighting}[]
\CommentTok{# duplicate object for testing}
\NormalTok{docs.lemma <-}\StringTok{ }\NormalTok{docs}

\CommentTok{# lemmatize the corpus (require textstem)}
\NormalTok{docs.lemma <-}\StringTok{ }\KeywordTok{tm_map}\NormalTok{(docs.lemma, lemmatize_strings)}
\KeywordTok{writeLines}\NormalTok{(}\KeywordTok{as.character}\NormalTok{(docs.lemma[[}\DecValTok{2}\NormalTok{]]))}
\end{Highlighting}
\end{Shaded}

\begin{verbatim}
## read warm tale camaraderie affection set wild beautiful savannah free illustrate kid book ginger giraffe use long neck save animal blaze forest fire follow jungle path meet yet another adventure
\end{verbatim}

\begin{center}\rule{0.5\linewidth}{0.5pt}\end{center}

\hypertarget{section-5-text-representation}{%
\section{Section 5: Text
Representation}\label{section-5-text-representation}}

\hypertarget{document-term-matrix-dtm}{%
\subsection{5-1: Document Term Matrix
(DTM)}\label{document-term-matrix-dtm}}

\textbf{Document Term Matrix} (or \textbf{DTM} for short) is a matrix
that lists all occurrences of words (column) in the corpus, by document
(row).

\begin{itemize}
\item
  A word that appears in a particular document will have its respective
  matrix entry in the corresponding row and column assign to 1, else 0.
\item
  A word that appears \emph{n} times in a document will be recorded as
  \texttt{n} in the respective matrix entry.
\end{itemize}

\hypertarget{example}{%
\paragraph{Example}\label{example}}

We have the two documents, \emph{Doc1} and \emph{Doc2}, with the
following content:

\begin{itemize}
\tightlist
\item
  \emph{Doc1}: goats are happy
\item
  \emph{Doc2}: goats are fat
\end{itemize}

The corresponding DTM will look like:

\begin{longtable}[]{@{}lllll@{}}
\toprule
& goats & are & happy & fat\tabularnewline
\midrule
\endhead
Doc1 & 1 & 1 & 1 & 0\tabularnewline
Doc2 & 1 & 1 & 0 & 1\tabularnewline
\bottomrule
\end{longtable}

DTM can become very huge, depending on the corpus. The dimension of the
DTM is the \textbf{\# of documents} multiplied by the \textbf{\# of
words in the corpus}. \textbf{Sparsity} often happens since majority of
words only appear in few documents.

\begin{Shaded}
\begin{Highlighting}[]
\CommentTok{# Document Term Matrix: DTM}

\CommentTok{# Create Document Term Matrix}
\NormalTok{dtm <-}\StringTok{ }\KeywordTok{DocumentTermMatrix}\NormalTok{(docs.lemma)}

\CommentTok{# View summary of document term matrix}
\NormalTok{dtm}
\end{Highlighting}
\end{Shaded}

\begin{verbatim}
## <<DocumentTermMatrix (documents: 10, terms: 163)>>
## Non-/sparse entries: 219/1411
## Sparsity           : 87%
## Maximal term length: 12
## Weighting          : term frequency (tf)
\end{verbatim}

\begin{Shaded}
\begin{Highlighting}[]
\CommentTok{# Inspect document term matrix}
\KeywordTok{inspect}\NormalTok{(dtm)}
\end{Highlighting}
\end{Shaded}

\begin{verbatim}
## <<DocumentTermMatrix (documents: 10, terms: 163)>>
## Non-/sparse entries: 219/1411
## Sparsity           : 87%
## Maximal term length: 12
## Weighting          : term frequency (tf)
## Sample             :
##                         Terms
## Docs                     adventure book find free fun illustrate little read
##   childrenstories_01.txt         1    1    1    1   1          1      0    1
##   childrenstories_02.txt         1    1    0    1   0          1      0    1
##   childrenstories_03.txt         0    1    0    1   1          1      2    1
##   childrenstories_04.txt         0    0    0    0   0          0      0    0
##   childrenstories_05.txt         0    1    0    0   0          0      0    0
##   childrenstories_06.txt         1    1    0    1   0          0      1    0
##   childrenstories_07.txt         0    0    0    0   1          0      0    1
##   childrenstories_08.txt         0    1    0    2   0          0      0    2
##   childrenstories_09.txt         0    0    2    1   0          0      1    1
##   childrenstories_10.txt         0    1    0    2   0          0      2    2
##                         Terms
## Docs                     story time
##   childrenstories_01.txt     0    0
##   childrenstories_02.txt     0    0
##   childrenstories_03.txt     1    1
##   childrenstories_04.txt     0    0
##   childrenstories_05.txt     0    0
##   childrenstories_06.txt     1    1
##   childrenstories_07.txt     2    1
##   childrenstories_08.txt     1    0
##   childrenstories_09.txt     2    0
##   childrenstories_10.txt     0    0
\end{verbatim}

\begin{Shaded}
\begin{Highlighting}[]
\CommentTok{# Inspect document term matrix by specifying rows and columns}
\KeywordTok{inspect}\NormalTok{(dtm[}\DecValTok{1}\OperatorTok{:}\DecValTok{5}\NormalTok{, }\DecValTok{11}\OperatorTok{:}\DecValTok{20}\NormalTok{])}
\end{Highlighting}
\end{Shaded}

\begin{verbatim}
## <<DocumentTermMatrix (documents: 5, terms: 10)>>
## Non-/sparse entries: 17/33
## Sparsity           : 66%
## Maximal term length: 10
## Weighting          : term frequency (tf)
## Sample             :
##                         Terms
## Docs                     free friend fun game grind hide hole illustrate just
##   childrenstories_01.txt    1      1   1    1     1    1    1          1    1
##   childrenstories_02.txt    1      0   0    0     0    0    0          1    0
##   childrenstories_03.txt    1      0   1    0     0    0    0          1    0
##   childrenstories_04.txt    0      0   0    0     0    0    0          0    0
##   childrenstories_05.txt    0      1   0    0     0    0    0          0    0
##                         Terms
## Docs                     kidsâ\200\231
##   childrenstories_01.txt       1
##   childrenstories_02.txt       0
##   childrenstories_03.txt       1
##   childrenstories_04.txt       0
##   childrenstories_05.txt       0
\end{verbatim}

\begin{center}\rule{0.5\linewidth}{0.5pt}\end{center}

\hypertarget{section-6-text-mining}{%
\section{Section 6: Text Mining}\label{section-6-text-mining}}

\hypertarget{words-frequency}{%
\subsection{6-1: Words Frequency}\label{words-frequency}}

When constructing the DTM, the corpus of text is converted into a
\textbf{mathematical object} that can be analyzed and manipulated using
quantitative techniques of matrix algebra.

To get the frequency of each word in the corpus, we can sum over all
rows based on the columns.

\begin{Shaded}
\begin{Highlighting}[]
\CommentTok{# get frequency of each word}
\NormalTok{freq <-}\StringTok{ }\KeywordTok{colSums}\NormalTok{(}\KeywordTok{as.matrix}\NormalTok{(dtm))}

\CommentTok{# check dimension of frequency (number of words/columns)}
\KeywordTok{length}\NormalTok{(freq)}
\end{Highlighting}
\end{Shaded}

\begin{verbatim}
## [1] 163
\end{verbatim}

\hypertarget{check-frequent-vs.-infrequent-words}{%
\subsubsection{Check Frequent vs.~Infrequent
Words}\label{check-frequent-vs.-infrequent-words}}

We can also sort the words (\texttt{freq}) in descending order based on
term count.

\begin{Shaded}
\begin{Highlighting}[]
\CommentTok{# Create sort order}
\NormalTok{ord <-}\StringTok{ }\KeywordTok{order}\NormalTok{(freq, }\DataTypeTok{decreasing =} \OtherTok{TRUE}\NormalTok{)}

\CommentTok{# Inspect most frequently occurring terms}
\NormalTok{freq[}\KeywordTok{head}\NormalTok{(ord)]}
\end{Highlighting}
\end{Shaded}

\begin{verbatim}
##      free      read      book     story    little adventure 
##         9         9         7         7         6         3
\end{verbatim}

\begin{Shaded}
\begin{Highlighting}[]
\CommentTok{# Inspect least frequently occurring terms}
\NormalTok{freq[}\KeywordTok{tail}\NormalTok{(ord)]}
\end{Highlighting}
\end{Shaded}

\begin{verbatim}
##     much     open    stand    truly validate      way 
##        1        1        1        1        1        1
\end{verbatim}

\hypertarget{terms-reduction}{%
\subsubsection{Terms Reduction}\label{terms-reduction}}

We can reduce the term in the DTM by specifying the following
parameters:

\begin{itemize}
\tightlist
\item
  number of documents the word appears in the corpus: 2 to 8 documents
\item
  length of words: 4 to 20 characters
\end{itemize}

\begin{Shaded}
\begin{Highlighting}[]
\CommentTok{# Create document term matrix with term reduction}
\CommentTok{# - Include only words that occur in 2-8 documents. }
\CommentTok{# - enforce lower & upper limit to the length of words (4-20 characters)}
\NormalTok{dtm.tr <-}\StringTok{ }\KeywordTok{DocumentTermMatrix}\NormalTok{(}
\NormalTok{  docs.lemma, }\DataTypeTok{control =} \KeywordTok{list}\NormalTok{(}
    \DataTypeTok{wordLengths =} \KeywordTok{c}\NormalTok{(}\DecValTok{4}\NormalTok{,}\DecValTok{20}\NormalTok{), }\DataTypeTok{bounds =} \KeywordTok{list}\NormalTok{(}\DataTypeTok{global =} \KeywordTok{c}\NormalTok{(}\DecValTok{2}\NormalTok{,}\DecValTok{8}\NormalTok{))}
\NormalTok{  )}
\NormalTok{)}

\NormalTok{dtm.tr}
\end{Highlighting}
\end{Shaded}

\begin{verbatim}
## <<DocumentTermMatrix (documents: 10, terms: 29)>>
## Non-/sparse entries: 81/209
## Sparsity           : 72%
## Maximal term length: 12
## Weighting          : term frequency (tf)
\end{verbatim}

\hypertarget{find-frequent-terms}{%
\subsubsection{Find Frequent Terms}\label{find-frequent-terms}}

Call function \texttt{findFreqTerm()}, then specify the DTM and filter
by \texttt{lowfreq\ =\ 5} (the output shows only words with 5 or more
occurrences in the corpus).

Note that the results is sorted alphabetically, not by frequency.

\begin{Shaded}
\begin{Highlighting}[]
\CommentTok{# Find frequent terms}
\KeywordTok{findFreqTerms}\NormalTok{(dtm.tr, }\DataTypeTok{lowfreq =} \DecValTok{5}\NormalTok{)}
\end{Highlighting}
\end{Shaded}

\begin{verbatim}
## [1] "book"   "free"   "read"   "little" "story"
\end{verbatim}

\hypertarget{terms-correlation}{%
\subsection{6-2: Terms Correlation}\label{terms-correlation}}

\textbf{Correlation} is a quantitative measure of the co-occurrence of
words in the corpus. The correlated terms can be identified by calling
\texttt{findAssocs()} in \texttt{tm} package, then specify the term of
interest and correlation limit.

\begin{Shaded}
\begin{Highlighting}[]
\CommentTok{# Find terms correlation}
\KeywordTok{findAssocs}\NormalTok{(dtm.tr, }\StringTok{"read"}\NormalTok{, }\FloatTok{0.5}\NormalTok{)}
\end{Highlighting}
\end{Shaded}

\begin{verbatim}
## $read
##         free        aloud         next childrenâ\200\231s 
##         0.80         0.79         0.79         0.51
\end{verbatim}

\hypertarget{example-calculation-of-correlation-score}{%
\paragraph{Example calculation of correlation
score}\label{example-calculation-of-correlation-score}}

Pearson Correlation Coefficient is defined as follows:

\[
\def\numer{n\sum_{i=1}^{n}{x_i y_i} - \sum_{i=1}^{n}{x_i}\sum_{i=1}^{n}{y_i}}
\newcommand{\denoms}[1]{\left(n\sum_{i=1}^{n}{{#1}_i^2} - \left(\sum_{i=1}^{n}{{#1}_i}\right)^2\right)}
r = \frac{\numer}{\sqrt{\denoms{x}\denoms{y}}}
\]

\begin{Shaded}
\begin{Highlighting}[]
\NormalTok{trydata <-}\StringTok{ }\KeywordTok{c}\NormalTok{(}
  \StringTok{""}\NormalTok{, }\StringTok{"word1"}\NormalTok{, }\StringTok{"word1 word2"}\NormalTok{, }\StringTok{"word1 word2 word3"}\NormalTok{, }
  \StringTok{"word1 word2 word3 word4"}\NormalTok{, }\StringTok{"word1 word2 word3 word4 word5"}
\NormalTok{)}

\NormalTok{trydtm <-}\StringTok{ }\KeywordTok{DocumentTermMatrix}\NormalTok{(}\KeywordTok{VCorpus}\NormalTok{(}\KeywordTok{VectorSource}\NormalTok{(trydata)))}
\NormalTok{trydtm}
\end{Highlighting}
\end{Shaded}

\begin{verbatim}
## <<DocumentTermMatrix (documents: 6, terms: 5)>>
## Non-/sparse entries: 15/15
## Sparsity           : 50%
## Maximal term length: 5
## Weighting          : term frequency (tf)
\end{verbatim}

\begin{Shaded}
\begin{Highlighting}[]
\KeywordTok{as.matrix}\NormalTok{(trydtm)}
\end{Highlighting}
\end{Shaded}

\begin{verbatim}
##     Terms
## Docs word1 word2 word3 word4 word5
##    1     0     0     0     0     0
##    2     1     0     0     0     0
##    3     1     1     0     0     0
##    4     1     1     1     0     0
##    5     1     1     1     1     0
##    6     1     1     1     1     1
\end{verbatim}

\begin{Shaded}
\begin{Highlighting}[]
\KeywordTok{findAssocs}\NormalTok{(trydtm, }\StringTok{"word1"}\NormalTok{, }\FloatTok{0.0}\NormalTok{)}
\end{Highlighting}
\end{Shaded}

\begin{verbatim}
## $word1
## word2 word3 word4 word5 
##  0.63  0.45  0.32  0.20
\end{verbatim}

\begin{center}\rule{0.5\linewidth}{0.5pt}\end{center}

\hypertarget{section-7-simple-graphics}{%
\section{Section 7: Simple Graphics}\label{section-7-simple-graphics}}

\hypertarget{histogram}{%
\subsection{7-1: Histogram}\label{histogram}}

\begin{Shaded}
\begin{Highlighting}[]
\NormalTok{freq.tr <-}\StringTok{ }\KeywordTok{colSums}\NormalTok{(}\KeywordTok{as.matrix}\NormalTok{(dtm.tr))}

\CommentTok{# Plot simple frequency histogram}
\CommentTok{# Create a data frame (consists of name of the column)}
\NormalTok{wordfreq <-}\StringTok{ }\KeywordTok{data.frame}\NormalTok{(}
  \DataTypeTok{term =} \KeywordTok{names}\NormalTok{(freq.tr), }\DataTypeTok{occurences =}\NormalTok{ freq.tr}
\NormalTok{)}

\CommentTok{# load ggplot2 package}
\KeywordTok{library}\NormalTok{(ggplot2)}
\end{Highlighting}
\end{Shaded}

\begin{verbatim}
## Warning: package 'ggplot2' was built under R version 3.6.3
\end{verbatim}

\begin{verbatim}
## 
## Attaching package: 'ggplot2'
\end{verbatim}

\begin{verbatim}
## The following object is masked from 'package:NLP':
## 
##     annotate
\end{verbatim}

\begin{Shaded}
\begin{Highlighting}[]
\CommentTok{#invoke ggplot(plot only terms more than 3 times, label x and y-axis using aes)}
\NormalTok{phisto<-}\KeywordTok{ggplot}\NormalTok{(}\KeywordTok{subset}\NormalTok{(wordfreq, freq.tr}\OperatorTok{>}\DecValTok{3}\NormalTok{), }\KeywordTok{aes}\NormalTok{(term, occurences))}
\CommentTok{#set the height of the bar using stat="bin" or "identity" ("identify" means the height is based on the data value mapped to y-axis)}
\NormalTok{phisto<-phisto }\OperatorTok{+}\StringTok{ }\KeywordTok{geom_bar}\NormalTok{(}\DataTypeTok{stat=}\StringTok{"identity"}\NormalTok{)}
\CommentTok{#specify that the x-axis text is at 45 degree angle and horizontally justified}
\NormalTok{phisto<-phisto }\OperatorTok{+}\StringTok{ }\KeywordTok{theme}\NormalTok{(}\DataTypeTok{axis.text.x=}\KeywordTok{element_text}\NormalTok{(}\DataTypeTok{angle=}\DecValTok{45}\NormalTok{, }\DataTypeTok{hjust=}\DecValTok{1}\NormalTok{))}
\CommentTok{#display histogram}
\NormalTok{phisto}
\end{Highlighting}
\end{Shaded}

\includegraphics{textminepipe_files/figure-latex/unnamed-chunk-24-1.pdf}

\hypertarget{word-cloud}{%
\subsection{7-2: Word Cloud}\label{word-cloud}}

\begin{Shaded}
\begin{Highlighting}[]
\CommentTok{# load wordcloud package}
\KeywordTok{library}\NormalTok{(wordcloud)}
\end{Highlighting}
\end{Shaded}

\begin{verbatim}
## Warning: package 'wordcloud' was built under R version 3.6.3
\end{verbatim}

\begin{verbatim}
## Loading required package: RColorBrewer
\end{verbatim}

\begin{Shaded}
\begin{Highlighting}[]
\CommentTok{# setting the seed before each plot to ensure consistent look for clouds}
\KeywordTok{set.seed}\NormalTok{(}\DecValTok{32}\NormalTok{)}

\CommentTok{# limit words by specifying min frequency}
\KeywordTok{wordcloud}\NormalTok{(}\KeywordTok{names}\NormalTok{(freq.tr), freq.tr, }\DataTypeTok{min.freq =} \DecValTok{3}\NormalTok{, }\DataTypeTok{scale =} \KeywordTok{c}\NormalTok{(}\FloatTok{3.5}\NormalTok{, }\FloatTok{0.25}\NormalTok{))}
\end{Highlighting}
\end{Shaded}

\includegraphics{textminepipe_files/figure-latex/unnamed-chunk-25-1.pdf}

\begin{Shaded}
\begin{Highlighting}[]
\CommentTok{# limit words by specifying min frequency (with color)}
\KeywordTok{wordcloud}\NormalTok{(}\KeywordTok{names}\NormalTok{(freq.tr), freq.tr, }\DataTypeTok{min.freq =} \DecValTok{3}\NormalTok{, }\DataTypeTok{scale =} \KeywordTok{c}\NormalTok{(}\FloatTok{3.5}\NormalTok{, }\FloatTok{0.5}\NormalTok{), }\DataTypeTok{colors =} \KeywordTok{brewer.pal}\NormalTok{(}\DecValTok{6}\NormalTok{, }\StringTok{"Dark2"}\NormalTok{))}
\end{Highlighting}
\end{Shaded}

\includegraphics{textminepipe_files/figure-latex/unnamed-chunk-25-2.pdf}

\end{document}
